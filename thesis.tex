\documentclass{article}
\usepackage[utf8]{inputenc}

% Graphics
\usepackage{graphicx}

% Miscellaneous
\usepackage{chngcntr} %Change counter

% Bibliography
\usepackage{apacite}
\bibliographystyle{apacite}

% Blank page without page number
\usepackage{afterpage}
\newcommand\blankpage{
    \null
    \thispagestyle{empty}
    \addtocounter{page}{-1}
    \newpage}
\setcounter{secnumdepth}{4}

% Begin document
\title{Master Thesis}
\begin{document}

\afterpage{\blankpage}

\begin{titlepage}
    \centering
 
    \vfill
    
    \includegraphics[width=\textwidth]{images/TiuLogo.eps}
    \vskip2cm
    {\huge
        Measuring the quality of texts digitized by Optical Character Recognition\\
        \large\bigskip
        by\\
        Mike Weltevrede (SNR 1257560)\\
        \vskip2cm
        A thesis submitted in partial fulfillment of the requirements for the degree of Master in Econometrics and Mathematical Economics.
        \vskip0.5cm
        Tilburg School of Economics and Management\\
        Tilburg University\\
        \vskip2cm
        Supervised by:\\
        dr. Otilia Boldea
		\vfill
        Date:\\
        \today
    }   
    \vfill
    \vfill
\end{titlepage}

\newpage

\tableofcontents

\newpage

% Acknowledgement
\section{Acknowledgements}\label{sec:acknowledgements}
% Your thank-you's fall into three categories: scientific/technical, financial and personal. Scientific and technical persons to consider thanking - your supervisor/s as well as other persons who helped you in the interpretation of your results, or in carrying out the practical aspects of your work, data collection, etc. In terms of finances, you can acknowledge that the research was supported by company X or through Dr. Y. In the last category you might any one that you feel helped you in your studies, in addition to friends or family who lent you support.

% Management Summary
\section{Management summary}\label{sec:management_summary}
% Potential readers will usually use this extract to determine whether or not your report is interesting enough to read would use this section. Therefore, it should be clear, concise and complete. It is easiest to write the summary after the report itself is written. A rough rule of thumb is to write an introductory paragraph of few lines, then one paragraph per chapter of your report and a few sentences summarizing your most important results, conclusions and recommendations. Total number of pages should not exceed two.

%Introduction
\section{Introduction}\label{sec:introduction}
% Contains literature review, your contribution briefly and how it is new relative to other studies, 1-3 main contributions of your thesis; each contribution should be backed up with one table to be found directly in the text.

\section{Problem description}\label{sec:problem_description}
% Description of problem and the models+ econometric methods you are using to solve it. No software descriptions please.

\section{Materials}\label{sec:material}
% Dataset and explanations

\section{Results}\label{sec:results}

\section{Conclusion}\label{sec:conclusion}
% Conclusions and future research

%Bibliography
\newpage

\bibliographystyle{plain}
\bibliography{Thesis}

\newpage\appendix

\section{Tables}\label{app:tables}

\end{document}
